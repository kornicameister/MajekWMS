\chapter{System WMS}
\label{c5:c5}

\section{Definicja systemu wspomagania gospodarki magazynowej}
	Głównym powodem stosowania systemów magazynowych w przedsiębiorstwach jest chęć optymalizacji
	operacji magazynowych oraz funkcjonowania samego magazynu i co za tym idzie uzyskania
	znaczących oszczędności. Zadaniem kompleksowego systemu wspomagającego zarządzanie magazynem jest dać możliwość
	nadzorowanie i kontrolowania procesów magazynowych, od momentu przyjęcia jednostki na magazyn,
	aż do momentu opuszczenia przez nią strefy wydań. Bardzo często zdarza się, że systemy tego rodzaju
	nie działają samodzielnie, a są ściśle zintegrowane z co najmniej jednym innym. Systemem tym na 
	ogół jest system należący do klasy ERP. Jest to wynikiem skomplikowania procesów
	magazynowych, które potrzebują odrębnych algorytmów, w które standardowy system informatyczny 
	działający w firmie, wyposażony nie jest, ale przedsiębiorstwie pragnie utrzymać kompleksową 
	kontrolę nad wszystkimi operacjami występującymi w systemie jako całości.\\
	
	Wyodrębnienie aspektów logistyki magazynowej było konieczne, ponieważ \textbf{systemy ERP}
	logistykę magazynową obsługują w zbyt wąskim zakresie, który kończy się na ustaleniu:
	\begin{itemize}
		\item stany magazynowe w ujęciu \textbf{ilościowym};
		\item stany magazynowe w ujęciu \textbf{jakościowym}.
	\end{itemize}
	Niestety wciąż pozostaje problem ujęcia fizycznej struktury zapasów, czyli cech takich jak
	\begin{itemize}
		\item logistyczne parametry opakowań;
		\item klasy miejsc składowania;
		\item oznaczenie miejsc w magazynie;
		\item oznaczenia poprzez kody kreskowe lub RFID 
			\footnote{Technika zastępująca tradycyjne czytniki laserowe, służy do przenoszenia
			informacji o produktach i możliwości ich odczytu przez czytniki oparte o technologie
			radiowe.} składowanych dóbr.
	\end{itemize}
	Tak więc z praktycznego punktu widzenia system klasy WMS jest rozbudowanym narzędziem
	wspierającym łańcuch dostaw na etapie magazynowanie i umożliwiającym monitorowanie
	oraz zarządzania składowanymi dobrami w sposób nie tylko jakościowy / ilościowy, ale i
	automatyczny dzięki utrzymaniu danych o fizycznej strukturze opakowań itp. 	
	
\section{Potrzeba korzystania z systemu WMS}
	Kompleksowość świadczenia usług logistycznych wymaga racjonalnego i szczegółowego podejścia
	do organizacji procesów magazynowych przy jednoczesnym zachowaniu możliwości zarządzaniami
	nimi w podejściu dającym pole do późniejszej rozbudowy i / lub re-definicji istniejącego
	rozwiązania. Argumentem przemawiającym za takim podejściem jest szerokie spektrum 
	produktów, materiałów jakie można znaleźć w strefach składowania dzisiejszych magazynów.
	Często dzieje się tak, że przykładowe przedsiębiorstwo produkcyjne posiada więcej niż jeden
	magazyn i to zlokalizowany w tym samym budynku. Niezbędne jest nie tyle obsłużenia
	wyjścia linii produkcyjnej i umieszczenia produktu gotowego na miejscach, do tego celu wyznaczonych.
	Wciąż pozostaje problem surowców, które również muszą być składowane i ewidencjonowane. Dzisiejsza
	logistyka i jej filozofia skłania się między innymi ku ciągłemu obniżaniu kosztów, bez obniżenia
	standardów jakości. \\
	
	Co jeszcze przemawia za system tej klasy, jaką jest \textbf{Warehouse Management System}. Otóż 
	wiele przedsiębiorstw wyszło już poza ramy lokalne, w jakich do tej pory się znajdowało. Rynek zbytu
	z lokalnego przeistoczył się w krajowy, czy też europejski. Zmiany jakie się z tym wiązały, to chociażby:
	\begin{itemize}
		\item wzrost ilości punktów dystrybucji gotowego produktu, koniecznych do zaopatrzenia;
		\item zdecentralizowanie magazynowych procesów logistycznych;
		\item zmniejszająca się wydolność magazynów opartych o niekompleksowego systemy
		obsługiwanego głównie przez człowieka.
	\end{itemize}
	Okazuje się, że dobrze skonfigurowany oraz gotowy na modyfikacje system wspomagający gospodarkę magazynową
	jest odpowiedzią na te wymagania. Ważne jest tutaj, że zarówno podejście autorskie, jak i uniwersalne są
	odpowiednio zbyt czasochłonne i zbyt sztywne. Dobry WMS charakteryzuję się wypośrodkowaniem oby tych 
	cech, w zależności od potrzeb klienta. \\
	
	\subsection{System WMS szyty na miarę}
	Należałoby się spodziewać, że \textbf{,,szyty na miarę''} oznacza w podejście skoncentrowane
	na stworzeniu systemu informatycznego od podstaw zgodnie ze specyfikacją otrzymaną od klienta.
	Mimo, że podejście to ma zalety, lepszym rozwiązaniem jest dostarczenie systemu, który 
	będzie można skonfigurować dla konkretnego odbiorcy. Właściwie sytuacja inna niż właśnie opisana,
	jest praktycznie niemożliwa, a systemu WMS działającego out-of-the-box\footnote{system gotowy do
	uruchomienia zaraz po instalacji} próżno szukać.\\

	W tym miejscu warto przywołać pojęcie \textbf{analizy logistycznej}. Pojęcie to opisuje w stopniu
	bardziej lub mniej dokładnym dwie proste czynności: \textbf{zdefiniowania potrzeb i problemów} 
	danego przedsiębiorstwa. W tym procesie powinny aktywnie uczestniczyć obie zainteresowane strony, 
	zarówno ta chcąca uzyskać działający i gotowy do użytku system, jak i firma dostarczająca
	rozwiązania. Ważne jest aby zdefiniować jak najwięcej zachować oraz cech magazynowania, jako całości,
	i uzgodnić je między stronami. 	
		
	Unikatowość procesów zachodzących w różnych magazynach, ich zmienny zakres generują popyt
	na program gotowy do spełniania tych funkcji. Podejście aspektowe \footnote{
		Podczas projektowania systemu, aplikacji nacisk kładzie się na rozpoznanie i
		zgrupowanie elementów odpowiedzialnych ze ten sam aspekt produktu końcowego	
	} wydaje się tutaj najbardziej trafnym wyborem. Gotowy produkt składa się z szeregów
	powiązanych ze sobą modułów, których funkcjonalność łatwo jest rozszerzać dodając nowe
	aspekty.

	\subsection{Samo rozwiązanie oparte o system nie wystarczy}
	Nieważne jak skomplikowany system można by wprowadzić w danym przedsiębiorstwie, okaże się
	bezużyteczny jeśli nie będzie wspierał łączności z urządzeniami zewnętrznymi. Mimo wciąż 
	postępującej automatyzacji procesów magazynów, udział czynnika ludzkiego nie pozostaje bez 
	znaczenia. I tak mamy na przykład strefę wydań, gdzie potrzeba magazyniera z umiejętnością
	obsługi wózka widłowego, aby przenieść towar na podstawiony samochód transportowy. 
	Magazynier po odebraniu, np. paletowej jednostki ładunkowej, musi ją zeskanować - czyli
	wprowadzić dane, że dokładnie ta paleta i znajdująca się na niej towary zostały załadowane.
	Taka kontrola nie byłaby możliwa dla systemu wspomagającego, jeśli nie zostałby on wyposażony
	w tę cenną umiejętność, jak obsługa urządzeń peryferyjnych. \\
	
	Nie tylko czytniki obsługiwane przez ludzi okazują się ważne. Jeśli spojrzeć na proces składowanie,
	jako nie na proces, gdzie człowiek jest tym, kto umieszcza daną jednostkę ładunkową w konkretnej lokacji
	magazynowej, ale na fazę kompletnie obsługiwaną przez maszynę, okaże się że rola pewnego
	rodzaju urządzeń jest również niezaprzeczalna. Magazynu wysokiego składu połączone muszę być w stanie
	informować system WMS o ruchach pozycji towarowych. Jest to ważne, ponieważ w tym wypadku, operator
	logistyczny może wydać polecenie wydania określonej liczby palet produktu X na bandę Y, ale to program
	komputerowy musi wiedzieć, skąd, jak i gdzie poprowadzić jednostki ładunkowe, aby stan
	końcowy był zgodny ze stanem oczekiwanym. Wspomniane urządzenia peryferyjne mają tutaj jeszcze jedno
	dodatkowe zastosowanie. Pozwalają śledzić, jak zamówienia jest kompletowane i jak poszczególne
	jego pozycje przesuwają się ku strefie kompletacji / wydań. 
	
\section{Funkcje systemu WMS}
	\subsection{Zarządzanie wieloma magazynami przez jeden interfejs}
	\subsection{Obszar magazynowy - jak zarządzać strukturą}
	\subsection{Miejsca magazynowy - kolejny poziom układu}
	\subsection{Zawartość magazynu - sposoby eksploracji}
	\subsection{Eksploracja magazynu}
	\subsection{Dokumenty logistyczne - jako element niezbędny w zarządzaniu}
	\subsection{Obsługa zamówień oraz zleceń}
	\subsection{Zamówienia dla dostawców}
	\subsection{Zlecenia od klientów}
	\subsection{Rezerwacja towarów}
	\subsection{Potwierdzanie dokumentów wydania}
	\subsection{Generacja ładunków}
	\subsection{Dostawy do magazynu}
		\subsubsection{Ewidencjonowanie dostaw zewnętrznych}
		\subsubsection{Ewidencjonowanie dostaw wewnętrznych}
	\subsection{Wysyłki z magazynu}
		\subsubsection{Ręczne planowanie wysyłek}
		\subsubsection{Automatyzacja planowania wysyłek}
	\subsection{Realizacja problemu kompletacji}
	\subsection{Problem operacji magazynowych}
	\subsection{Wsparcia dla planowania transportu i/lub spedycji}
