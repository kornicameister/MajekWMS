\chapter{Algorytmy alokacji produktów}
\label{ca:allocationAlgorithms}

	\section{Opis algorytmu}
		Zastosowany w aplikacji algorytm alokacji produktów opiera się na prostym 
		przypisaniu największej ilości palet do największego dostępnego miejsca. 
		Jest to algorytm, który został oparty o rozwiązania znane jako 
		\textbf{strategia przydziału pierwszego dopasowania}.
		
		\textbf{Strategia przydziału pierwszego dopasowania} jest metodą ciągłą,
		stosowaną na przykład przy przydziale wolnych zasobów dyskowych. Nadrzędną zasadą
		jest sekwencyjne przeszukanie zasobów i sprawdzeniu, czy odpowiadają one 
		potrzebom.
		
		\textbf{Algorytm alokacji zasobów} opiera się na tej samej zasadzie. Pobiera
		on listę wszystkich dostępnych stref, które są posortowane malejąco po dostępnej
		ilości pól odkładczych. Z drugiej strony produkty, które mają zostać
		rozlokowane w magazynie posortowane są malejąca po ilości palet, każdej z pozycji.
		Możliwe są dwie sytuacje:
		\begin{itemize}
			\item \textbf{dopasowanie pełne} - ilość palet danego produktu jest mniejsza lub
			równa dostępnemu rozmiarowi danej strefy;
			\item \textbf{dopasowanie częściowe} - ilość palet danego produktu jest większa od ilości
			dostępnych miejsc odkładczych. W tym wypadku alokowane jest możliwie, jak najwięcej palet
			danej pozycji, a reszta zostaje przekierowana do kolejnej ze stref;
		\end{itemize}

	\section{Kod źródłowy algorytmy alokującego przy przyjęciu magazynowym}
		\begin{code}
			\inputminted[
				linenos=true,
				obeytabs=true,
				tabsize=2,
				fontsize=\small,
				firstline=116, 
				lastline=153,
		        numbersep=2pt
			]{java}{../application/src/main/java/org/kornicameister/wms/model/logic/controllers/InvoiceProductController.java}
		\end{code}	