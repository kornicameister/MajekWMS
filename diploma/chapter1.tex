\chapter{Wstęp}
\label{c1:c1}

\section{Uzasadnienie wyboru tematu}
	Kilkanaście lat temu w Polsce systemy WMS praktycznie nie istniały. Przedsiębiorstwa
	produkcyjne opierały zarządzanie procesami magazynowymi o system papierowych dokumentów
	oraz wiedzy magazynierów. Od kilku, a nawet kilkunastu lat, można jednak obserwować, że
	systemy klasy WMS są coraz częściej używane,a firmy z nich korzystające zapomniały 
	o powszechnie panującym micie dotyczącym nieopłacalności takiego rozwiązania z uwagi na wysokie 
	koszty wprowadzenia, zapominając przy tym, że nadrzędną cechą systemów WMS jest
	przede wszystkim \emph{obniżenie kosztów obsługi magazynowej}, co praktycznie przekłada się na wzrost 
	zysków.
	
	Trudno dzisiaj sobie wyobrazić nowoczesny magazyn bez urządzeń takich jak \textbf{wózki widłowe}, 
	\textbf{regały}, \textbf{pasy transmisyjne}, itp. Niestety wciąż w bardzo wielu magazynach praca 
	opiera się głównie na systemie notesów, kartek, które 
	są ekwiwalentem baz danych, a długopis w ręku magazyniera odpowiednikiem terminala radiowego. Nie 
	można się tutaj oprzeć wrażeniu, że takie podejście do spraw zarządzania rozmieszczeniem towarów
	ma dużą podatność na występowanie błędów mogących wyniknąć z rzeczy pozornie błahych, jakimi są
	zwolnienie lekarskie lub zgubienie jednego z dokumentów magazynowych.
	
	Prawidłowo zaprojektowany, i co ważniejsze wykorzystywany, system WMS może przyczynić się do:
	znaczącego podniesienia jakości usług świadczonych przed przedsiębiorstwo, do zwiększenia
	szybkości realizacji zamówień i szybkości reakcji na zmiany w najbliższym otoczeniu
	przedsiębiorstwa. Kolejnym punktem, który przemawia za systemami wspomagającymi 
	zarządzanie magazynami są koszty, a dokładnie ich obniżenie, z uwagi na zmianę w podejściu 
	do administracji zapasami. Ostatecznie przedsiębiorstwo, które zdecydowało się na wprowadzenie
	systemu WMS z pewnością podniesienie jakość swoich produktów, zminimalizuje ryzyko
	uszkodzeń oraz stanie się bardziej konkurencyjne na rynku.
	\pagebreak	
\section{Cel i zakres pracy}
	Głównym celem pracy jest analiza kluczowych funkcji, którymi powinien cechować
	się system klasy WMS. Z tego powodu szczególna uwaga zostanie poświęcona takim
	zagadnieniom jak utrzymanie, zarządzanie oraz uaktualnianie informacji o 
	składowanych towarach lub produktach, czy też materiałach. Jest to z pewnością
	podstawowa cecha takiego systemu, która odróżnia go od systemu ERP, gdzie nacisk
	kładzie się na zarządzania towarami w kontekście bardziej jakościowym, niż
	ilościowym. Ważne stają się takie operacje jak kompletacje czy też alokacje towarów
	lub bardziej ogólnie wszystkie procedury odnoszące się do fizycznych przepływów
	wewnątrz magazynów.
	 
	Celem pracy jest również rozpoznanie najczęściej spotykanych problemów związanych
	z wprowadzaniem oraz działaniem systemów WMS. Analiza biznesowa przedsiębiorstwa
	jest jednym z pierwszych etapów, które poprzedzają właściwe wdrożenie
	gotowego rozwiązania, co potwierdza jej wagę w kontekście eliminacji przyszłych
	problemów.
	
	Intencją było stworzenie aplikacji, która demonstrowałaby część z szeregu funkcji,
	które można znaleźć w systemach wspomagających gospodarkę magazynową, a także dała
	możliwość rozpoznania problemów wynikających z tego, związanych zarówno z samymi
	funkcjami logistycznymi, a także kwestiami czysto programistycznymi, wynikającymi
	z mapowania fizycznych elementów struktury magazynowej, 
	takich jak produkty, strefy wejściowe czy też kompletacyjne,
	na obiekty właściwe językom programowania.
	
