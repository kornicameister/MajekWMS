\chapter{Systemy WMS jako system informatyczny}
\label{c3:c3}

\section{Czym jest magazyn w łańcuchu logistycznym, jaką pełni funkcje?}
	Systemy wspomagające zarządzania magazynem dotyczą, jak sama nazwa wskazuje, magazynów.
	Czym jest magazynem w tym wypadku ?
	\begin{quotation}
		Magazyn jest jednostką funkcjonalno-organizacyjną, przeznaczoną do magazynowania
		dóbr materialnych (zapasów) w wyodrębnionej przestrzeni, budowli magazynowej, według ustalonej
		technologii, wyposażoną w odpowiednie urządzenia i środki techniczne, zarządzaną i obsługiwaną
		przez zespół ludzi, wyposażonych w odpowiednia umiejętności.
	\end{quotation}
	\referenceSource{PN-N-01800:1984}{norm_warehouse_defintion}
	
	Wobec tego magazyn nie jest jedynie statycznym bytem łańcuchów logistycznych, ale jego integralną
	częścią bez której żadne przedsiębiorstwo, w szczególności produkcyjne, nie mogłoby sprawnie
	funkcjonować. Ogół czynności, czyli \textbf{magazynowanie}. jest tym, co związane jest z manipulowaniem
	zapasami oraz ich przechowywanie. Czynności te mogą odnosić się zarówno do początkowej fazy procesów gospodarczych
	u producenta danego dobra, znajdować się pośrodku łańcucha logistycznego jako magazyn centralny, bądź u klienta
	końcowego, hurtownika. Niezależnie od miejsca występowania magazynowanie zdefiniowane jest przez cztery
	podstawowe punkty: 
	\begin{itemize}
		\item przyjęcie
		\item składowania
		\item kompletacja
		\item wydanie
	\end{itemize}
	\cite{PZMW}\cite{PL_FM}
	
	\subsection{Magazynowania jako proces}
		\paragraph{Przyjęcie} jest wydarzeniem zewnętrznym, które rozpoczyna cały ciąg czynności związanych z przetwarzaniem go.
		Warto w tym miejscu zaznaczyć, że faza przyjęcia nie jest równoważna z wprowadzaniem danych o przyjmowanych
		dobrach do magazynu i uaktualnieniu stanów. Zanim to nastąpi należy rozładować produkty oraz poddać je kontroli
		zarówno jakościowej i ilościowej \cite{PL_FM}.
			\subparagraph{Kontrola ilościowa} czyli sprawdzeniu stanu faktycznego z oczekiwanym, zgodnie ze stanem np. dokumentu przewozowego. 
			Sprawdzenie to przyjmuje różne formy, które są najczęściej uzależnione od rodzaju artykułu. Różnicę łatwo wskazać
			poprzez porównanie produktów takich jak zboża (\textit{produkt sypki}) czy bele papieru. W pierwszym przypadku 
			kontrola ilościowa prawdopodobnie przebiegnie z użyciem wagi samochodowej, a w drugim przypadku zostanie zliczona
			ilość sztuk beli.  
			\subparagraph{Kontrola jakościowa} wykazuje z drugiej strony bezpośrednie i silne powiązanie z wymaganiami prawnymi bądź
			różnorodnymi normami odnoszącymi się do danego rodzaju dóbr. Dla większości produktów jest to kontrola
			wzrokowa w poszukiwaniu uszkodzeń mechanicznych, wycieków itp. Niemniej warto nadmienić, że w przypadku
			artykułów, których natura jest szkodliwa dla ludzi, zwierząt, taka kontrola może potrwać niejednokrotnie
			więcej niż jeden dzień i wymagać przeprowadzenia badań laboratoryjnych.
		\paragraph{Składowania} polega na usystematyzowanym umieszczaniu dóbr materialnych, które najczęściej przyjmują
		postać jednostek ładunkowych (\textit{palety EURO}), w przestrzeni magazynu. Czynność ta, jako będąca właściwie
		esencją każdego z magazynów, jest bardzo rozbudowana, przez co wspomagana przez różnego rodzaju urządzania, począwszy
		od wózków widłowych, pasów transmisyjnych po regały dowolnego typu . Z tego powodu strefy składowania dzielą się na:
		\begin{enumerate}
			\item \textbf{strefy składowania} - wydzieloną przestrzeń służącą do 
			przechowywanie zapasów w urządzeniach bądź piętrzenia w stosy
			\item \textbf{strefy manipulacyjnej} - wszystkie drogi manipulacyjne 
			oraz przejazdowe nie będące przestrzenią składowania
		\end{enumerate}	
		W tym miejscu należy zwrócić uwagę na konieczność uwzględnienia takich właściwości fizycznych / chemicznych, które
		mają bezpośrednie przełożenie na warunki, w jakich należy składować dane dobra. Do takich parametrów należą
		chociażby:
		\begin{enumerate}
			\item temperatura, wilgotność powietrza
			\item właściwości trujące, wybuchowe materiałów itp\footnote{Decydują o tym wymogi prawne \cite{ustawa_flamableMaterials}}
			\item odporność na zgniatanie \footnote{szczególnie ważna w przypadku piętrzenia w stosy}
		\end{enumerate}			
		Wynika z tego także, że składowania jest ważnym elementem magazynowania, ponieważ to od tego, czy zostanie
		wykonane poprawnie, zależy czy towar zostanie uszkodzony bądź nie \cite{EWSS}.
	\paragraph{Kompletacja}
	\paragraph{Wydanie}
	
	\subsection{Podział magazynu na strefy - cele i korzyści}
	
\section{Definicja systemu wspomagania gospodarkę magazynową - WMS}
\section{Potrzeba korzystania z systemu WMS, kiedy jest uzasadniona}
\section{Funkcje systemu WMS w logistyce magazynowej}