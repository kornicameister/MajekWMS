\chapter{Konfiguracja Hibernate na podstawie pliku XML}
\label{ca:hibernatecfg}
	
	\section{Opis konfiguracji}
		Plik konfiguracyjny dla Hibernate, tj. najczęściej plik XML \textit{hibernate.cfg.xml}
		jest nieodłącznym elementem, bez którego nie można by w ogóle uruchomić narzędzia. 
		Istnieje oczywiście możliwość ustawienia odpowiednich opcji programowo, ale 
		nie jest to rozwiązania uniwersalne i ograniczone koniecznością podmiany
		skompilowanych plików Javy\footnote{pliki o rozszerzeniu *.jar}. \\
		Plik konfiguracyjny powinien zawierać:
		\begin{itemize}
			\item dane do ustanowienia połączenia z silnikiem bazy danych;
			\item dane do zarządzania wykonaniem kwerend;
			\item dane o mapowanych klasach, każda taka klasa musi znaleźć się w omawianym pliku.
		\end{itemize}
	
	\section{Plik konfiguracyjny hibernate.cfg.xml}
		\begin{code}
			\inputminted[
				linenos=true,
				fontsize=\small,
				obeytabs=true,
				tabsize=1,
				resetmargins=true,
		        numbersep=2pt
			]{xml}{../application/src/main/resources/hibernate.cfg.xml}
		\end{code}	