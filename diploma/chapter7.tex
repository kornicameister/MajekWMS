\chapter{Podsumowanie}
\label{c7:c7}
	Celem pracy była analiza cech oraz funkcji, którymi powinien się charakteryzować system
	wspomagający gospodarkę magazynową, dzięki zaprojektowaniu od podstaw oraz 
	przygotowaniu aplikacji demonstrującej przykładowy system WMS. Praktycznie każde przedsiębiorstwo, które obecnie działa na rynku,
	opiera swoje funkcjonowanie o różnego rodzaju narzędzia informatyczne wspomagające procesy wytwórcze, zaopatrzeniowe, dystrybucyjne
	bądź magazynowe. Jest więc rzeczą oczywistą, że potrzebują one aplikacji, z jednej strony na tyle kompleksowych, 
	obejmujących zasięgiem działania przedsiębiorstwo jako całość, a z drugiej strony dających dostęp
	do kolejnych części składowych firmy, z których każda odpowiedzialna jest za wykonanie konkretnych zadań,
	
	W kontekście aplikacji zrealizowana została następująca funkcjonalność: 
	możliwość defi\-niowania nowej firmy oraz magazynu doń przypisanego,
	możliwość utworzenia oraz modyfikowania logicznej struktury magazynu,
	możliwość tworzenia nowych klientów, zarówno dostawców oraz odbiorców,
	możliwość tworzenia wydań oraz przyjęć magazynowych, które dają przypisać się do dowolnego klienta,
	z zastrzeżeniem, że wydania odnoszą się jedynie do odbiorców, a przyjęcia do dostawców,
	uproszczony system statystyk oraz tabel opisujących aktualny stan magazynu. 	

	Dobry system wspomagający zarządzanie gospodarką magazynową powinien
	cechować się automatyzacją pewnych procesów oraz dawać możliwość wykonania
	podstawowych operacji, których celem oraz wynikiem byłaby zmiana fizycznej oraz logicznej struktury
	opisującej rozmieszczenie materiałów, produktów lub towarów w magazynie.
	Niemniej jednak udało się dostarczyć metod, które implementowałyby taką funkcjonalność, co spowodowało,
	że przygotowana aplikacja mogłaby służyć jako system wspomagający gospodarkę magazynową w 
	małym przedsiębiorstwie.
	
	Aplikacja, przygotowana w ramach projektu inżynierskiego, dostarcza ułamka funkcjonalności
	programów wspierających gospodarkę magazynową, wykorzystywanych w przedsiębiorstwach. 
	Narzędzia funkcjonujące w firmach dostarczają zdecydowanie większych możliwości, co przekłada się na poziom
	ich skomplikowania. Tego typu aplikacje budowane są wtedy pod kątem danego przedsiębiorstwa, gdzie brana jest
	pod uwagę specyfika i preferencje organizacji. Jest to szczególnie ważne, ponieważ na dzisiejszym
	rynku, ważne jest nie tyle szybkość i precyzja, co umiejętność dostosowania się
	do ciągle zmieniającego się otoczenia biznesowego.
	
	Oczywiście mimo, że aplikacja porusza część z funkcji systemu WMS, możliwy jest jej rozwój. Przede wszystkim
	wykonalne jest wprowadzenie ewentualnego usprawnienia dotyczącego uogólnienia sposobu modelowanie fizycznej 
	struktury magazynu o osobiste prefe\-rencje klienta, tak aby odpowiadała ona jego wymaganiom. 
	Kolejnym elementem byłoby rozszerzenie obecnych	możliwości zarządzania fizycznymi przepływami towarów o dodanie sposobu, który pozwoliłby
	na wprowadzenie do systemu produktów, materiałów, które jeszcze nie były na stanie magazynowym (przyjęcia nowego towaru).
	Warto wspomnieć o autoryzacji użytkowników uwzględniającej w obecnej formie jedynie sprawdzenie danych
	logowanie pod kątem autentyczności loginu oraz hasła. Wprowadzenie poziomów dostępów opartych o role,
	gdzie do jednej roli, może być przypisanych wielu użytkowników, z pewnością
	dałoby większe możliwości kontroli i pozwoli na tworzenie użytkowników końcowych z zakresem uprawnień,
	pozwalających jedynie przeglądać treści aplikacji, przy jednoczesnym wyłączeniu ich modyfikacji.
	Ostatecznie można by zająć się usprawnieniem modułu odpowiedzialnego za statystyki magazynowe o dodanie funkcji pozwalających
	na generowanie podsumowań zależnych od informacji, jakie będzie chciał uzyskać użytkownik systemu. 