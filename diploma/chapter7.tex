\chapter{Podsumowanie}
\label{c7:c7}
	Zrealizowana została następująca funkcjonalność:
	\begin{itemize}
		\item możliwość definiowania nowej firmy oraz magazynu doń przypisanego;
		\item możliwość utworzenia oraz modyfikowania logicznej struktury magazynu;
		\item możliwość tworzenia nowych klientów, zarówno dostawców oraz odbiorców;
		\item możliwość tworzenia wydań oraz przyjęć magazynowych, które można było przypisać do
		dowolnego klienta, z zastrzeżeniem, że wydania odnoszą się jedynie do odbiorców, a przyjęcia
		do dostawców;
		\item uproszczony system statystyk oraz tabel opisujących aktualny stan magazynu;
	\end{itemize}
	
	Projekt, będący praktyczną częścią pracy inżynierskiej, okazał się być trudniejszy do zrealizowania,
	niż początkowo było to przewidywane. Pojawiające się trudności można zaklasyfikować i przypisać 
	do następujących grup:
	\begin{itemize}
		\item związane z wymianą danych serwer - klient;
		\item związane z dostarczenie funkcji systemu WMS; 
	\end{itemize}		
	
	\paragraph{Wymiana danych serwer - klient} okazała się być kluczowym elementem całego programu, 
	od którego zależały wszystkie inne komponenty. Podstawowym problemem okazało się więc stworzenie
	takiego modelu danych, których charakteryzowałby się spójnością między częścią kliencką oraz serwerową,
	przy jednoczesnym dostarczeniu uniwersalnego interfejsu zdolnego do obsługi czterech rodzajów
	zadań: \textbf{Dodaj, Usuń, Zmień, Pobierz}. 
	
	\paragraph{Funkcje systemu WMS} mając znaczenie fundamentalne dla działania programu okazały
	się być niemniej wymagające. Dobry system wspomagający zarządzanie gospodarką magazynową powinien
	cechować się automatyzacją pewnych procesów oraz powinien dawać możliwość wykonania
	podstawowych operacji, których celem oraz wynikiem byłaby zmiana fizycznej oraz logicznej struktury
	opisującej rozmieszczenie materiałów, produktów lub towarów w magazynach.
	Niemniej jednak udało się dostarczyć metod, które implementowały by taką funkcjonalność, co spowodowało,
	że przygotowana aplikacja mogłaby służyć jako system wspomagający gospodarkę magazynową, w 
	małym przedsiębiorstwie.