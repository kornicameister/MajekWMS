\chapter{Wstęp}
\label{c1:c1}

% p1 %%%%%%%%%%%%%%%%%%%%%%%%%%%%%%%%%%%%%%%%%%%%%%%%%%%%%%%%%%%%%%%%%%%%%%%%%%%%
\section{Uzasadnienie wyboru tematu}
	\hspace{10pt}Kilkanaście lat temu w Polsce nikt nie słyszał o systemach WMS, mimo że na świecie
	obecne były już dużo dłużej. Jeśli ktoś jednak już słyszał, to nie był zainteresowany
	wprowadzaniem takiego systemu z uwagi na powszechnie panujący mit o nieopłacalności takiego
	rozwiązania z uwagi na wysokie koszty wprowadzenia, zapominając przy tym, że nadrzędną cechą systemów WMS jest
	przede wszystkim \emph{obniżenie kosztów obsługi magazynem}, co praktycznie przekłada się na wzrost 
	zysków.
	
	Trudno dzisiaj sobie wyobrazić nowoczesny magazyn bez urządzeń takich jak \textbf{wózki widłowe}, 
	\textbf{regały}\footnote{Warto w tym miejscu nadmienić, że regały dzielą się bezpośrednio na stałe,
	przejezdne i specjalizowane}, \textbf{pasy transmisyjne}, itp. \\ 
	Niestety wciąż w bardzo wielu magazynach praca opiera się głównie na systemie notesów, kartek, które 
	są ekwiwalentem baz danych, a długopis w ręku magazyniera odpowiednikiem terminala radiowego. Nie 
	można się tutaj oprzeć wrażeniu, że takie podejście do spraw zarządzania rozmieszczeniem towarów
	ma dużą podatność na występowanie błędów mogących wyniknąć z rzeczy pozornie błahych, jak
	\begin{itemize}
		\item zwolenianie lekarskie
		\item zgubienie dokumentu dowolnego rodzaju
	\end{itemize}
	
	Prawidłowo zaprojektowany, i co ważniejsze wykorzystywany, system WMS, może przyczynić się do:
	\begin{itemize}
		\item znaczącego podniesienia jakości usług, świadczonych przez przedsiębiorstwo
		\item zwiększenia szybkości realizacji zamówień
		\item obniżenia kosztów obsługi magazynowej
		\item podniesienia bezpieczeństwa produktu
	\end{itemize}
% p1 %%%%%%%%%%%%%%%%%%%%%%%%%%%%%%%%%%%%%%%%%%%%%%%%%%%%%%%%%%%%%%%%%%%%%%%%%%%%

% p2 %%%%%%%%%%%%%%%%%%%%%%%%%%%%%%%%%%%%%%%%%%%%%%%%%%%%%%%%%%%%%%%%%%%%%%%%%%%%
\section{Cel i zakres pracy}
	\hspace{10pt}Głównym celem pracy jest analiza \emph{systemów WMS} realizacji 
	podstawowych funkcji. Z tego powodu szczególna uwaga zostanie poświęcona zagadnieniom
	takim jak:
	\begin{itemize}
		\item utrzymanie informacji o strukturze klientów
		\begin{itemize}
			\item dostawców
			\item odbiorców
		\end{itemize}
		\item utrzymanie informacji o towarach
		\begin{itemize}
			\item rozłożenie towarów w magazynie
			\item wydania oraz przyjęcia
			\item podstawowe algorytmy alokacji
		\end{itemize}
		\item podgląd raportów
		\begin{itemize}
			\item o fakturach klientów
			\item strukturze magazynów
			\item aktualnych stanach magazynowych
		\end{itemize}
	\end{itemize}
	
	Ponadto zrealizowane zostały niżej wymienione punkty:
	\begin{itemize}
		\item analiza algorytmów alokacji produktów w magazynie
		\item analiza i rozpoznanie problemów w magazynach
		\item rozpoznanie warunków wymaganych do wprowadzenia systemu
		\item analiza czynności przed wprowadzeniem systemu WMS
	\end{itemize}
% p2 %%%%%%%%%%%%%%%%%%%%%%%%%%%%%%%%%%%%%%%%%%%%%%%%%%%%%%%%%%%%%%%%%%%%%%%%%%%%

% p3 %%%%%%%%%%%%%%%%%%%%%%%%%%%%%%%%%%%%%%%%%%%%%%%%%%%%%%%%%%%%%%%%%%%%%%%%%%%%
\section{Przegląd i zakres literatury}
% p3 %%%%%%%%%%%%%%%%%%%%%%%%%%%%%%%%%%%%%%%%%%%%%%%%%%%%%%%%%%%%%%%%%%%%%%%%%%%%
