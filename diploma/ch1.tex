\chapter{Wstęp}
\label{c1:c1}

% @todo - left-to-right
\section{Opis tematu}

	\hspace{10pt} Dawno już minęły czasy, kiedy do zarządzania magazynem wystarczyła obecność jednego
	doświadczonego magazyniera nie mającego do dyspozycji komputera czy tabletu. Można
	to krótkie zdanie podsumować bardzo prosto: \textit{Te czasy już minęły}. 
	
	Dzisiejsza obsługa magazynów to zadanie kompleksowe i zdecydowanie przekraczające
	ludzkie możliwośći. W tym miejscu wchodzą systemu informatyczne specjalnie projektowane, 
	aby wspomagać procesy decyzyjne, automatyzować zadania i dostarczać narzędzi analitycznych.
	
	Jakie korzyści daje dokładnie zastosowanie \emph{systemu wspomagającego zarządzanie gospodarką magazynową}.
	Z pewnością zaletami są:
	\begin{itemize}
		\item przyspieszenie dynamiki łańcucha dostaw
		\item ograniczenie czynnika ludzkiego, potrzebnego jedynie na etapie planowania
		\item dostęp do stanów magazynów w czasie rzeczywistym
		\item gromadzenie informacji historycznych
		\item możliwość automatyzacji procesów wydań i przyjęć
	\end{itemize}

% p1 %%%%%%%%%%%%%%%%%%%%%%%%%%%%%%%%%%%%%%%%%%%%%%%%%%%%%%%%%%%%%%%%%%%%%%%%%%%%
\section{Uzasdnienie wyboru tematu}

	\hspace{10pt} Często zdarza się, że przedsiębiorstwo generuje poważne straty, dochodząc do momentu
	kiedy, kierownictwo uświadamia sobie, że nie potrafi znaleźć odpowiedzi na pytanie, co właściwie
	się stało, że system, działający do tej pory perfekcyjnie i zapewniający zyski, nagle stał się
	piętą Achillesową. Niejednokrotnie przyczyną takie stanu rzeczy są przestarzałe metody zarządzania
	produkcją, magazynowaniem, zbyt wysokie koszty magazynowania. 
	
	
	Prawidłowo zaprojektowany, i co ważniejsze wykorzystywany, system WMS, może przyczynić się do:
	\begin{itemize}
		\item znaczącego podniesienia jakości usług, świadczonych przez przedsiębiorstwo
		\item zwiększenia szybkości realizacji zamówień
		\item obniżenia kosztów obsługi magazynowej
		\item podniesienia bezpieczeństwa produktu
	\end{itemize}
% p1 %%%%%%%%%%%%%%%%%%%%%%%%%%%%%%%%%%%%%%%%%%%%%%%%%%%%%%%%%%%%%%%%%%%%%%%%%%%%

% p2 %%%%%%%%%%%%%%%%%%%%%%%%%%%%%%%%%%%%%%%%%%%%%%%%%%%%%%%%%%%%%%%%%%%%%%%%%%%%
\section{Cel i zakres pracy}

Lorem ipsum
% p2 %%%%%%%%%%%%%%%%%%%%%%%%%%%%%%%%%%%%%%%%%%%%%%%%%%%%%%%%%%%%%%%%%%%%%%%%%%%%